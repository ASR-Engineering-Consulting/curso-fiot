\documentclass{beamer}
\usepackage[utf8]{inputenc}
\usetheme{Berkeley}
\usecolortheme{seahorse}

\title{Fundamentos de Internet das Coisas}
\subtitle{Apresentação da disciplina}

\AtBeginSection[]
{
	\begin{frame}
	\frametitle{Sumário}
	\tableofcontents[currentsection]
\end{frame}
}

\begin{document}

\frame{\titlepage}

\begin{frame}
\frametitle{Sumário}
\tableofcontents
\end{frame}

\section{Objetivos}

\begin{frame}
\frametitle{Objetivos da disciplina}
Objetivo principal
\begin{itemize}
	\item Apresentar os principais conceitos relacionados ao desenvolvimento de aplicações voltadas para a área de Internet das Coisas (IoT), partindo das tecnologias e arquiteturas disponíveis com o intuito de implementar protótipos funcionais.
\end{itemize}
\end{frame}

\begin{frame}
\frametitle{Objetivos da disciplina}
Objetivos específicos
\begin{itemize}
	\item Guiados pelos exercícios desenvolvidos durante a disciplina, os estudantes irão:
	\begin{itemize}
		\item explorar diversas tecnologias de IoT, 
		\item desenvolver conhecimentos sobre a experiência do usuário com tais tecnologias e 
		\item tomar decisões inteligentes relacionadas ao tratamento de dados. 
	\end{itemize} 
	
	\item Além disso, os resultados destas experiências poderão:
	\begin{itemize}
		\item viabilizar novos negócios com o intuito de promover e desenvolver a economia disruptiva, 
		\item facilitar a escolha de tecnologias e um melhor entendimento dos aspectos de segurança relacionados à área.
	\end{itemize}
\end{itemize}
\end{frame}

\begin{frame}
\frametitle{Objetivos da disciplina}
Objetivos práticos
\begin{itemize}	
	\item Outro objetivo vem a ser o de habilitar os estudantes para trabalharem com:
	\begin{itemize}
		\item microcontroladores como Arduino e ESPs, 
		\item microcomputadores como o Raspberry Pi, 
		\item diversos sensores e dispositivos atuadores, 
		\item interfaces para comunicação sem fio do tipo Bluetooth e Wifi, 
		\item entre outras tecnologias que permitam a transmissão de dados pela Internet.
	\end{itemize}
\end{itemize}
\end{frame}

\begin{frame}
\frametitle{Objetivos da disciplina}
Objetivos finais
\begin{itemize}	
	\item Tendo em vista que a Internet das Coisas se baseia na troca de dados pela Internet com propósito de tomada de decisão, o tratamento e análise dos dados requer uma abordagem especial considerando a informação esperada além do custo energético e computacional para a aquisição e transmissão da mesma. 
\end{itemize}
\end{frame}

\section{Justificativa}
\begin{frame}
	\frametitle{Justificativas da disciplina}
	\begin{itemize}
		\item O mundo atual está interligando as pessoas e coisas através de aplicações que estão \textbf{conectadas pela Internet}. 
		
		\item Passamos a viver um paradigma no qual é difícil pensar sobre \textbf{o que não estará conectado} em um futuro próximo. 
		
		\item O uso de IoT nos leva à concepção de \textbf{cidades inteligentes}. Estas proporcionam às pessoas uma vida facilitada através de aplicações que tornam o dia a dia mais prático por meio de ambientes mais \textbf{interligados e responsivos}. 
	\end{itemize}
\end{frame}

\begin{frame}
\frametitle{Justificativas da disciplina}
\begin{itemize}	
	\item A maior dificuldade para implementação de soluções no ramo da Internet das Coisas passa a ser então:
	\begin{itemize}
		\item a união de conhecimentos sobre software e hardware;
		\item juntamente com noções de redes de computadores, circuitos digitais e analógicos, segurança;
		\item e ferramentas úteis que proporcionem o desenvolvimento de tecnologias de maneira prática e dinâmica.
	\end{itemize} 
\end{itemize}
\end{frame}

\section{Conteúdo}

\begin{frame}
\frametitle{Conteúdo da disciplina}
\begin{itemize}
	\item Introdução à Internet das Coisas e seu panorama atual: Uma visão da área do ponto de vista acadêmico e empresarial. 
	
	\item Obtenção de dados em IoT: Sensores (temperatura, luminosidade, presença), APIs e dados abertos.
	
	\item Conectando coisas: Microcontroladores (Arduino), Microprocessadores (Raspberry Pi), Web Services, Cloud Services.
	
	\item Protocolos de rede e Internet para aplicações em IoT: Bluetooth, WiFi, MQTT, CoAP.
	
	\item Tratamento de dados de sensores: Filtros, fusão de sensores, e aprendizado de máquina.
	
	\item Segurança e privacidade em nível de IoT.
	
	\item Desenvolvimento de aplicações para cidades inteligentes. 
\end{itemize}
\end{frame}


\section{Bibliografia}

\begin{frame}
\frametitle{Bibliografia em inglês}
\begin{itemize}
	\item Keary, Mae. \textbf{The Internet of Things} (The MIT Press Essential Knowledge Series). Online Information Review, 2016.
	
	\item Kranz, Maciej. \textbf{Building the Internet of Things: Implement New Business Models, Disrupt Competitors, Transform Your Industry}. John Wiley \& Sons, 2016.
	
	\item Rifkin, Jeremy. \textbf{The zero marginal cost society: The Internet of Things, the collaborative commons, and the eclipse of capitalism}. St. Martin's Press, 2014.
\end{itemize}
\end{frame}

\begin{frame}
\frametitle{Bibliografia em inglês}
\begin{itemize}
\item Kellmereit, Daniel; Daniel Obodovski. \textbf{The silent intelligence: the Internet of Things}. DnD Ventures, 2013.

\item Friess, Peter. \textbf{Internet of Things - Global Technological and Societal Trends From Smart Environments and Spaces to Green ICT}. River Publishers, 2011.

\item McRoberts, Michael. \textbf{Beginning Arduino}. New York.: Apress, 2010.
\end{itemize}
\end{frame}

\begin{frame}
\frametitle{Bibliografia em português}
	\begin{itemize}		
		\item Javed, Adeel (Autor); Adas, Cláudio José (Tradutor). \textbf{Criando Projetos com Arduino Para a Internet das Coisa}s. Novatec. 2017
		
		\item Oliveira, Sergio. \textbf{Internet das Coisas com ESP8266, Arduino e Raspberry Pi}. São Paulo: Novatec (2017).
	\end{itemize}
\end{frame}

\begin{frame}
\frametitle{Bibliografia especial}
\begin{itemize}	
	\item Atzori, L., Iera, A., \& Morabito, G. (2010). \textbf{The Internet of Things: A survey}. Computer Networks, 54(15), 2787–2805. \url{http://doi.org/10.1016/j.comnet.2010.05.010}
	
	\item Voas, Jeffrey. \textbf{Networks of ‘things’}. NIST Special Publication 800 (2016): 183. \url{http://dx.doi.org/10.6028/NIST.SP.800-183}
\end{itemize}
\end{frame}

\frame{\titlepage}

\end{document}
